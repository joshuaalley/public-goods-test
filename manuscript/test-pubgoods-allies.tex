\documentclass[12pt]{article}

\usepackage{fullpage}
\usepackage{graphicx, rotating, booktabs} 
\usepackage{times} 
\usepackage{natbib} 
\usepackage{indentfirst} 
\usepackage{setspace}
\usepackage{grffile} 
\usepackage{hyperref}
\usepackage{adjustbox}
\setcitestyle{aysep{}}


\singlespace
\title{
\textbf{Reassessing the Public Goods Theory of Alliances}
	}
\author{Joshua Alley\footnote{Graduate Student,
Department of Political Science, Texas A\&M University.}}
\date{{\normalsize \today}}

\bibliographystyle{apsr}

\begin{document}

\maketitle 

\doublespace

\begin{abstract}
The public goods theory of alliances exerts substantial influence on scholarship and policy, especially through its claim that smaller alliance participants can free-ride on larger partners because alliances face collective action problems. 
Prior statistical tests of free-riding suffer from model specification and generalizability problems, however, so there is little reliable and general evidence about this key prediction.
In this study, I address those limitations with a new test of the free-riding hypothesis. 
Using data on 204 alliances from 1919 to 2007, I examine how often states with low economic weight in an alliance decrease military spending, and states with high economic weight increase military spending. 
I find little evidence to support this expression of the free-riding hypothesis. 
The findings may reflect the importance of bargaining and exchange among alliance members. 
\end{abstract} 

\newpage


%----------------------------------
\section{Introduction}



\citet{OlsonZeckhauser1966} argue that international alliances generate a collective action problem. 
According to their theory, security from an alliance is a public good, so smaller alliance participants ``free ride'' on the contributions of larger members. 
Free-riding is reflected by disproportionate allocations of resources to defense, where smaller alliance members spend a lower share of their national income on the military, relative to larger partners.
In this paper, I test this key implication of the public goods model in a way that addresses model specification and generalizability issues in previous research. 
I use the public goods logic to predict that states with high economic weight in an alliance will increase percentage changes in military spending, and states with low economic weight will decrease military spending.
Then I employ a Bayesian model to estimate the association between economic weight and percentage changes in military spending for 204 alliances. 
I find little evidence for the predictions of the public goods model. 


Why undertake another test of the public goods model? 
Although academic theory has progressed substantially since 1966, the public goods model retains an important place in discourse about alliances.
The public foods model and related modifications \citep{Sandler1993} have a salient place in scholarly debate e.g.,\citep{Walt1990, Mearsheimer1994, Goldstein1995, SandlerHartley2001, Garfinkel2004, Walt2009, Norrlof2010, Barrett2010, PluemperNeumayer2015}. 


In addition, policy and popular discussions of alliances employ collective action ideas.
Pundits and American policymakers often refer to allied ``free-riding.'' 
%Treating alliances as a public good beset by collective action problems generates concern that the United States is ``being taken advantage of.'' 
%US policymakers often use free-riding to criticize lackluster allied defense expenditures.  
For example, Barack Obama complained in 2016 that ``Free riders aggravate me'' and US allies ``have to pay your fair share.'' 
Donald Trump has implied the United States would not protect allies who spend too little on defense. 
Such complaints and exhortations go back as far as the Eisenhower administration \citep{Lanoszka2015}.


% this is the spot for Caverley's frame: debates about benefits and burdens of US Alliances 
Increasing great power competition makes assessing Olson and Zeckhauser's argument even more important. 
Competing visions of US grand strategy hinge in part on the explanatory power of the public goods model. 
Advocates of retrenchment and ``restraint'' use the public goods logic to claim that US allies free-ride, so the United States should withdraw from many alliances \citep{Preble2009, Posen2014}. 
Others assert that alliances do not provide a public good and the benefits of alliance participation outweigh the costs \citep{Brooksetal2013, BrandsFeaver2017}. 


% So what am I doing here?: revisiting is necessary. 
The prominence of the public goods model belies a major problem.
53 years after the publication of ``An Economic Theory of Alliances,'' there is little reliable and general evidence about Olson and Zeckhauser's prediction that small states are prone to free-ride. 
Here, I address that empirical gap in the literature. 


My approach addresses two key limitations in existing tests of the public goods logic.
First, many empirical estimates of free-riding within alliances have a model specification problem from the dependent variable.
Olson and Zeckhauser use military spending as a share of GDP to measure contributions to the alliance, and GDP to measure state size.
This approach is widely emulated, but it creates a problem because GDP is present on both sides of any correlation.
Changes in GDP shift the defense burden, and this deterministic component affects correlation and regression estimates.\footnote{
See the appendix for a formal demonstration of this claim. 
When GDP changes, the defense burden remains constant only if military spending also changes in such a way that defense spending's share of GDP remains the same.}  


One notable paper addresses the dependent variable specification problem, but may not produce general findings. 
\citet{PluemperNeumayer2015} examine how growth in military spending by North Atlantic Treaty Organization (NATO) members responds to changing US and Soviet spending.
They demonstrate that NATO members are unresponsive to US and Soviet military spending, and present this as evidence of free riding.
\citet{PluemperNeumayer2015} find no correlation between NATO member size and the extent of free-riding, however, which they argue contradicts Olson and Zeckhauser.
The emphasis on NATO in this paper brings me to the second limitation: a lack of generalizability. 


% Huge (over)emphasis on NATO 
NATO is the epicenter of free-riding discussions. 
Following Olson and Zeckhauser's emphasis on military spending as a share of GDP, accusations of free-riding emphasize that NATO members have lower defense burdens than the United States. 
Scholars, pundits and policymakers have spent decades arguing over how well the public goods model applies to NATO, e.g. \citep{Pryor1968, SandlerForbes1980, Palmer1990, HiltonVhu1991, Boyer1993, GatesTerasawa1992, SandlerHartley2001, Lanoszka2015, PluemperNeumayer2015, KimSandler2019}.


% So what is the problem here?: it's mostly NATO
Most studies of the public goods model focus on NATO, but NATO is a difficult case for making general conclusions. 
NATO is exceptionally large, durable and capable. 
There are only seven tests of the public goods model outside of NATO, all of which examine a few alliances \citep{Russett1970, Starr1974, Reisinger1983, Thies1987, ConybeareSandler1990, OnealWhatley1996, Siroky2012}. 
Six of these studies include GDP in the independent and dependent variable, so they suffer from the aforementioned specification problem.
The seventh study uses case comparisons.
This leaves a need for a general examination of the public goods logic. 



% I'm not the first one to address this theory: first comprehensive empirical evidence
Using a Bayesian model that estimates the association between economic weight and percentage changes in military spending, I find little evidence that small states are more inclined to free-ride than their larger allies.
These findings will not surprise skeptics of the public goods model of alliances \citep{Palmer1990, GatesTerasawa1992, SandlerHartley2001, Norrlof2010, NiouZeigler2019}.
Some theoretical skepticism is based on inconsistent correlations between GDP and defense burdens. 
These correlations do not provide reliable evidence, however. 
Without a reliable and general test, theoretical revisions of a parsimonious public goods model may be premature.


The paper proceeds as follows.
First, I summarize the public goods theory of alliances and use it to derive observable implications of free-riding.
Then, I describe the model and results. 
The final section assesses implications for scholarship and policy. 



\section{Free-Riding in Alliances}

% I do not test their exact proposition 
To identify observable implications of free-riding, I use the public goods argument to derive predictions about economic size and military spending and alliances.
This theoretical exercise does not critique the public goods model.
Instead, I take the public goods logic as given, but apply an alternative measure of defense effort. 
Following \citet{PluemperNeumayer2015} I use percentage changes in military spending to measure defense effort.
I then assess the role of relative state size using economic weight. 


% summarize argument
Why might alliances suffer from a collective action problem?
The aggregate military capability of an alliance provides security for members and states contribute by investing in their military.
Because an alliance cannot exclude members without undermining its purpose, alliance security is a public good. 
Alliance members receive security regardless of their individual contribution.\footnote{The marginal costs and benefits of participation depend in part on alliance size, but Olson and Zeckhauser's model shows free-riding even in a bilateral alliance.}
Thus, states have incentives to rely on their alliance partners and reduce their own military expenditures.  

 
Olson and Zeckhauser expect that larger members of the alliance will bear a higher defense burden, because these states value security from the alliance more.
Small alliance members rely on larger partners for security and reduce their defense burdens.
As a result, smaller states free-ride on larger alliance participants. 
Moreover, smaller states have greater bargaining leverage, because a large state cannot credibly threaten to reduce their contribution and has ``relatively less to gain than its small ally from driving a hard bargain'' \citep[pg. 274]{OlsonZeckhauser1966}. 


% Economic weight as another way to get at O+Z's key comparion
Olson and Zeckhauser compare alliance members using economic resources, specifically GDP.
Economic weight within an alliance, or each state's share of total allied GDP, is a related way to conceptualize differences in state size.\footnote{Different states may be large or small depending on the alliance.} 
Using economic weight facilitates state size comparisons across diverse alliances. 
A greater share of total economic resources in an alliance gives a state more economic weight and increases their potential military spending contribution. 


Because Olson and Zeckhauser expect that economic weight shapes defense expenditures. 
Inasmuch as larger states value security from an alliance more, alliance participation will increase their investment in military capability.
Smaller members can free-ride and lower military spending. 
Therefore, alliance participation will increase the military spending of large states, and decrease military spending by small states. 


\begin{quote}
\textsc{Hypothesis 1}: For states with a large share of the total GDP in an alliance, alliance participation will increase annual percentage changes in military spending. 
\end{quote}


\begin{quote}
\textsc{Hypothesis 2}: For states with a small share of the total GDP in an alliance, alliance participation will decrease annual percentage changes in military spending.
\end{quote}


I assess the two hypotheses by estimating the association between economic weight and military spending across 204 alliances. 
Each alliance has an economic weight parameter to capture the consequences of alliance participation. 
I code economic weight such that a positive economic weight parameters implies more military spending for large members and less spending by small members. 
A negative economic weight parameter implies that larger members spend less om the military, and small members spend more. 
To assess the public goods model, I examine how many alliances have a positive economic weight parameter.  
 

% Clarify that both predictions approximate the O+Z logic
Though my predictions use slightly different variables, the changes facilitate a reliable and general empirical test.
I now test the hypotheses in a sample of all non-microstates from 1919 to 2007.\footnote{Limited GDP data makes constructing economic weights for each alliance difficult before 1919. I also omit some alliances after 1919 due to a lack of GDP data.}
State-year observations are the unit of analysis.


\section{Testing the Public Goods Logic}


For each of the 204 alliances that promise military support,\footnote{ATOP offensive and defensive treaties \citep{Leedsetal2002}.} I estimate a parameter measuring the association between economic weight and military expenditures. 
This model examines alliance free-riding using evidence from many treaties.\footnote{I also make similar inferences by regressing percentage changes in military spending on a state's average weight in their alliances. See the appendix.}
Bayesian estimation regularizes estimates with many parameters, so I fit the following model using STAN \citep{Carpenteretal2016}.


The model starts with state-year percentage changes in military spending $y_{it}$, transformed with an inverse hyperbolic sine.
I model this variable using a t-distribution with degrees of freedom $\nu$ to account for heavy tails.
The expected value of military spending $\mu_{it}$ depends on a constant $\alpha$, state and year varying intercepts, and control variables $\mathbf{X_{it}} \beta$.\footnote{See the appendix for a full description of all the variables in the model.} 
$\sigma$ captures unexplained variation in $y$. 

\begin{equation}
y_{it} \sim student_t(\nu, \mu_{it}, \sigma) 
\end{equation}

\begin{equation}
\mu_{it} = \alpha + \alpha^{st} + \alpha^{yr} + \mathbf{X_{it}} \beta + \mathbf{Z_{it}} \gamma
\end{equation}


The $\mathbf{Z_{it}} \gamma$ term captures the impact of economic weight in alliances.  
$\textbf{Z}$ is a matrix of state participation in alliances--- columns are alliances, rows are state-year observations.  
If a state is part of an alliance, the corresponding value in $\textbf{Z}$ depends on their share of total GDP in the alliance. 
Small alliance members are assigned a value of negative 1 if their economic weight is less than the median GDP share of .5 in bilateral alliances, or less than the median GDP share of .07 in multilateral alliances.\footnote{The distribution of economic weight is very different in bilateral and multilateral alliances, so I use different medians.}
Large alliance members have a value of positive one in $\textbf{Z}$ if their economic weight is above those thresholds. 
If a state is not part of the alliance, the corresponding matrix element is zero. 
Multiplying a positive $\gamma$ by negative one for small states will lead to negative military spending growth.
Multiplying the same economic weight parameter by positive one will increase military spending growth for large states. 


$\textbf{Z}$ is a quasi-spatial approach to capturing the impact of participation in multiple alliances.
In this model, alliance participation affects military spending through economic weight.  
The $\gamma$ parameters capture the correlation between economic weight in an alliance and military spending. 


$\gamma$ is a vector of 204 alliance-specific economic weight parameters.  
Because \textbf{Z} contains each state's share of allied GDP, these coefficients estimate the association between economic weight and military spending.\footnote{This assumes symmetric effects across small and large states, which I relax somewhat in the appendix by using a weighted coding of $\textbf{Z}$} 
When a state is not in an alliance, the corresponding $\gamma$ is multiplied by zero, and has no impact. 


Each alliance has a separate impact on military spending.
While these alliance parameters are distinct, they have a common prior distribution.
Partial pooling estimates the dispersion of the alliance parameters from the data, so the prior for $\gamma$ is normally distributed with mean $\theta$ and variance $\sigma_{all}$. 
$\theta$ is the mean hyperparameter of the alliance coefficients and each $\gamma$ deviates from $\theta$ based on a variance hyperparameter $\sigma_{all}$.
Every alliance estimate holds the impact of other treaties constant. 
A positive $\gamma$ implies that alliance members with more economic weight have higher percentage changes in military spending, as Hypothesis 1 predicts. 
    


\subsection{Results} 


The public goods model would expect many positive economic weight parameters. 
Because I employed Bayesian modeling, each $\gamma$ has a posterior distribution.\footnote{See the appendix for a full summary of priors, convergence and model fit. I also show that the model recovers known parameters from simulated data.} 
I focus interpretation on the posterior mean and 90\% credible intervals.\footnote{I use 90\% credible intervals because inferences around 95\% intervals are less stable.}
The posterior mean is the expected value of $\gamma$, while the credible intervals capture uncertainty around that estimate.  


There are no alliances with a clear positive economic weight parameter.
\autoref{fig:alliance-coefs-year} plots the $\gamma$ parameter for each alliance against the start year of the treaty.
Points mark the posterior mean. 
The error bars encapsulate the 90\% credible interval.


\begin{figure}[htbp]
	\centering
		\includegraphics[width=0.95\textwidth]{alliance-coefs-year.pdf}
	\caption{Estimated association between economic weight and defense spending growth in 204 defensive and offensive alliances from 1919 to 2007. Positive estimates match the predictions of Hypotheses 1 and 2. Points represent the posterior mean and the error bars cover the 90\% credible interval. The dashed line marks the posterior mean of the $\theta$ parameter, which is the average association between economic weight and percentage changes in military spending.}
	\label{fig:alliance-coefs-year}
\end{figure}


No alliances have a uniformly positive 90\% credible interval. 
Most credible intervals are consistent with effects ranging from -0.02 to 0.015. 
Only 13 of the 204 alliances have a positive posterior mean. 


% Can cut this if needed. 
%As \autoref{fig:alliance-coefs-year} shows, partial pooling of the $\gamma$ parameters regularizes many estimates. 
%$\theta$ has a posterior mean of -0.003. 
%The economic weight parameters are tightly clustered around that mean value. 
%Under these circumstances, it is difficult to distinguish most estimates from zero, even as partial pooling shares information across coefficients and reduces uncertainty. 


To further examine whether increasing a state's share of allied GDP leads to higher defense spending, I simulated the effect of changing economic weight on percentage changes in military spending. 
In the simulated data, I used the full posteriors of the intercept $\alpha$, all the $\beta$ coefficients, and one $\gamma$ parameter. 
I selected the economic weight parameter with the most positive posterior mass, which is the \emph{best case alliance} for Hypotheses 1 and 2. 
I set the state-level variables at their median or modal value and changed economic weight from -1 to 1. 


In \autoref{fig:pred-change-share}, I summarize predicted changes in military spending at the two economic weight values. 
In this figure, the point marks the mean and the error bars summarize the 90\% credible interval. 
There is limited evidence that larger alliance participants increase military spending. 

\begin{figure}[htbp]
	\centering
		\includegraphics[width=0.95\textwidth]{pred-change-share.pdf}
	\caption{Predicted percentage changes in military spending for a simulated state with low or high shares of total allied GDP. Hypotheses 1 and 2 would expect a positive difference between the high and low economic weight scenarios. Points mark the median value, and the error bars summarize the 90\% credible interval. The difference estimate captures the effect of moving from low to high economic weight.}
	\label{fig:pred-change-share}
\end{figure}


The $\gamma$ estimates also produce inferences about individual alliances.
The estimated $\gamma$ for NATO offers no support for the public goods theory of alliances. 
The posterior mean of this parameter is -.003, and it has 81\% \emph{negative} posterior mass.  
More evidence points to higher spending growth by small NATO members, relative to the US and sometimes the UK and France. 
This finding corroborates the size result of \citet{PluemperNeumayer2015}, but NATO members may still spend less on the military thanks to allied capability \citep{GeorgeSandler2017}.
Among other alliances, the Arab League (ATOPID 3205) has 78\% positive posterior mass, which is the most of any alliance. 
The alliance between the US and South Korea (ATOPID 3240) has 85\% negative posterior mass. 


\section{Conclusion}

% Add paragraph summarizing results
Few alliances see large differences in military spending by economic weight. 
Although Olson and Zeckhauser's model is parsimonious, two predictions from it do not apply to most alliances. 
Scholars can turn to several existing explanations for why this is the case. 


There are two possible explanations for the findings above. 
One is that alliances generate exchanges between alliance members, so we should focus on sources of leverage in intra-alliance bargaining \citep{Morrow1991, Norrlof2010, Brooksetal2013, Johnson2015, Kim2016}. 
Alternatively, the extent to which security from an alliance is a public good may vary with factors like technology \citep{SandlerHartley2001}. 


Policymakers and pundits should reassess the way they use ``free-riding'' to describe alliance politics. 
Free-riding is inextricable from a public goods understanding of alliances.
But if the public goods model has little empirical support, free-riding is an inaccurate description of reduced defense effort by alliance participants.  
Instead reduced defense effort could reflect cheap-riding on allied capability, where the pooling of military resources generates efficiency gains in defense spending. 


Scholars should not abandon the public goods model in international politics, however.   
Using alliances to understand collective action problems in international organizations more generally \citet[pg. 266-7]{OlsonZeckhauser1966} may be inappropriate, but collective action can apply to other international organizations. 
Still, with little evidence of free-riding based on economic weight, policymakers and scholars should be cautious about relying on a public goods model to understand alliance politics.  



\singlespace


\bibliography{../../../MasterBibliography} 





\end{document}

